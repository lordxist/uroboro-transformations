\documentclass[xcolor=svgnames]{beamer}

\usepackage[utf8]    {inputenc}
\usepackage[T1]      {fontenc}
\usepackage[english] {babel}

\usepackage{amsmath,amsfonts,graphicx,listings}
\usepackage{beamerleanprogress}
\usepackage{bussproofs}
\usepackage{fixltx2e}
\usepackage[numbers]{natbib}

\usepackage{lmodern}

\usepackage{tikz}
\usetikzlibrary{calc}

% tikzmark command, for shading over items
\newcommand{\tikzmark}[1]{\tikz[overlay,remember picture] \node (#1) {};}

\newcommand{\redbox}[2]{\tikz[overlay,remember picture]{\draw<#2>[draw=red,thick,fill opacity=0.2] ($(#1)+(-0.5,0.4)$) rectangle ($(#1)+(6,-0.2)$);}}

\lstdefinelanguage{Uroboro}
{
  % list of keywords
  keywords={
    data,
    codata,
    function,
    where
  },
  keywordstyle=\bfseries,
  comment=[l]{--} % l is for line comment
}

\lstset{columns=fullflexible,basicstyle=\ttfamily,language={Uroboro}}

\lstdefinestyle{base}{
  language={Uroboro},
  moredelim=*[is][\alert<.(1)->]{|}{|},
  moredelim=**[is][\only<+(1)>{\color{red}}]{@}{@}
}

\frenchspacing

\title
  [Automatic Program Transformations ...\hspace{2em}]
  {Automatic \only<-4>{{\color<3>{red}Program Transformations}}\\ \only<-4>{{\color<4>{red}for a Language}} with \only<-4>{{\color<2>{red}Copattern Matching}}}

\author
  [Julian Jabs]
  {\only<1>{Julian Jabs}}

\date
  {\only<1>{October 19, 2015}}

\begin{document}

\begin{frame}<1>[label=title]
  \titlepage
\end{frame}

\begin{frame}{Overview}
\tableofcontents
\end{frame}

\section
  {Motivation}

\begin{frame}
  {Motivation}

  \begin{itemize}
    \item Important feature of many functional programming languages: \textbf{first-class functions}

    \item \textbf{First-order} programs (without first-class functions) VS. \textbf{higher-order} programs (with first-class functions)
  \end{itemize}
\end{frame}

\begin{frame}
  {Motivation}

  \begin{itemize}
    \item First-order programs: certain properties beneficial for machine processing, recognizable as abstract machines

    \item Higher-order programs: beneficial for human understandability

    \item Want to find and automatize transformations between first-order and higher-order programs which preserve the program's semantics

    \item Concepts connected to this: \textbf{patterns} and \textbf{copatterns}
  \end{itemize}
\end{frame}

\section
  {Background}

\subsection
  {Patterns and copatterns}

\againframe<2>{title}

\begin{frame}[fragile]
  {Patterns and copatterns}

  \begin{block}{Data and patterns}
    \begin{lstlisting}[style=base, gobble=4]
    @data Nat where@
      @zero(): Nat@
      @succ(Nat): Nat@

    @function add(Nat, Nat): Nat where@
      @add(zero(), n)  = n@
      @add(succ(m), n) = succ(add(m, n))@
    \end{lstlisting}
  \end{block}
\end{frame}

\begin{frame}[fragile]
  {Patterns and copatterns}

  \begin{block}{Codata and copatterns\citep{abel13copatterns}}
    \begin{lstlisting}[style=base, gobble=4]
    @codata Stream where@
      @Stream.head(): Nat@
      @Stream.tail(): Stream@

    @function repeat(Nat): Stream where@
      @repeat(n).head() = n@
      @repeat(n).tail() = repeat(n)@
    \end{lstlisting}
  \end{block}
\end{frame}

\begin{frame}
  {Patterns and copatterns}

  \begin{block}{Coverage}
    A set of equations \textbf{covers} the function when the coverage for their lhss can be derived with \textbf{variable splits} and \textbf{result splits}.

  \begin{prooftree}
    \AxiomC{}
    \RightLabel{C\textsubscript{Head}}
    \UnaryInfC{$fun \lhd | (fun(\overline{x}))$}
  \end{prooftree}

  \begin{prooftree}
    \AxiomC{$fun \lhd | ~ Q ~ (q^\tau)$}
    \RightLabel{C\textsubscript{Des}}
    \UnaryInfC{$fun \lhd | ~ Q ~ (q.des(\overline{x^{des}}))_{des \in Dess_\tau}$}
  \end{prooftree}

  \begin{prooftree}
    \AxiomC{$fun \lhd | ~ Q ~ (q(x^\tau))$}
    \RightLabel{C\textsubscript{Con}}
    \UnaryInfC{$fun \lhd | ~ Q ~ (q[x \mapsto con(\overline{y^{con}})])_{con \in Cons_\tau}$}
  \end{prooftree}

  (Adapted from \citet{abel13copatterns})
  \end{block}
\end{frame}

\begin{frame}
  {Patterns and copatterns}
\underline{Example:}

Suppose (co)data types \texttt{Nat} and \texttt{Stream} as before. 

Consider function \texttt{f(Nat): Stream}.

Start with \texttt{f(x)} (C\textsubscript{Head}).
    \begin{itemize}
      \item Variable splits (C\textsubscript{Con}):

        \only<1-2>{\texttt{f(\alert<2>{x})}} \only<3->{$\substack{\texttt{f(\alert<3>{zero()})} \\ \only<3-4>{\texttt{f(\alert<3>{succ(\alert<4>{x})})}} \only<5->{\substack{\texttt{f(succ(zero()))} \\ \texttt{f(succ(succ(x)))}}}}$}

      \item Result splits (C\textsubscript{Des}):

        \only<-7>{\texttt{\alert<7>{f(x)}}} \only<8->{$\substack{\texttt{f(x).head()} \\ \only<8-9>{\texttt{\alert<9>{f(x).tail()}}} \only<10->{\alert<10>{\substack{\texttt{f(x).tail().head()} \\ \texttt{f(x).tail().tail()}}}}}$}

      \item Variable \underline{and} result splits:

        \only<-11>{\texttt{\alert<11>{f(x)}}} \only<12->{$\substack{\texttt{f(x).head()} \\ \only<12-13>{\texttt{f(\alert<13>{x}).tail()}} \only<14->{\alert<14>{\substack{\texttt{f(zero()).tail()} \\ \texttt{f(succ(x)).tail()}}}}}$}
    \end{itemize}



\end{frame}

\subsection
  {(De|Re)functionalization}

\againframe<3>{title}

\begin{frame}
  {(De|Re)functionalization}

  \begin{block}{Defunctionalization \citep{reynolds72definitional}}
    From \textbf{higher-order} to \textbf{first-order} programs.

    Introduces a \textbf{data type} for functions:

    \textbf{One constructor per first-class function}.
  \end{block}

  \begin{block}{Refunctionalization (original form) \citep{danvy09refunctionalization}}
    \textbf{Left inverse} of defunctionalization.
  \end{block}
\end{frame}

\begin{frame}[fragile]
  {(De|Re)functionalization}

  \begin{block}{Defunc. example}
    \begin{lstlisting}[style=base, gobble=4]
    @function all(Nat -> Bool, List): Bool where@
      @all(f, x:xs) = (f x) && all(f, xs)@

    @|function even: Nat -> Bool| where@
      @even(zero())  = true()@
      @even(succ(n)) = not(even(n))@

    @|function odd: Nat -> Bool| where@
      @odd(n) = not(even(n))@
    \end{lstlisting}
  \end{block}
  \pause
\end{frame}

\begin{frame}[fragile]
  {(De|Re)functionalization}

  \begin{block}{Defunc. example, defunc'ed}
    \begin{lstlisting}[style=base, gobble=4]
    @data Fun where@
      @|even(): Fun|@
      @|odd(): Fun|@

    @function all(Fun, List): Bool where@
      @all(f, x:xs) = apply(f, x) && all(f, xs)@

    @function apply(Fun, Nat): Bool where@
      @apply(even(), zero())  = true()@
      @apply(even(), succ(n)) = 
       not(apply(even(), n))@
      @apply(odd(), n) = not(apply(even, n))@
    \end{lstlisting}
  \end{block}
  \pause
\end{frame}

\begin{frame}[fragile]
  {(De|Re)functionalization}

  \textbf{Refunc'ing} the last program gives back the original example.

  \textbf{But:} With the original form of refunctionalization, one can only refunc. programs in the image of defunctionalization.

  \begin{block}{Counterexample}
    \begin{lstlisting}[escapechar=!]
    data Fun where
      fun1(): Fun
      fun2(): Fun

    function apply1(Fun, X): Y where
      ...
    
    function apply2(Fun, X): Y where
      ...
    \end{lstlisting}
  \end{block}
\end{frame}

\begin{frame}[fragile]
  {(De|Re)functionalization}
  \begin{block}{Counterexample, refunc'ed}
    \begin{lstlisting}[escapechar=!]
    codata Fun where
      Fun.apply1(X): Y
      Fun.apply2(X): Y

    function fun1(): Fun where
      ...
    
    function fun2(): Fun where
      ...
    \end{lstlisting}
  \end{block}
\end{frame}

\begin{frame}[fragile]
  {(De|Re)functionalization}

Solution of \citet{rendel15automatic}: 

\textbf{``Generalize first-class functions to arbitrary codata''}

\begin{block}{First-class functions as codata}
    \begin{lstlisting}[style=base, gobble=4]
    @codata Fun where@
       @Fun.apply(Nat): Bool@

    @function all(Fun, List): Bool where@
      @all(f, x:xs) = f.apply(x) && all(f, xs)@

    @function even(): Fun where@
      @even.apply(zero())  = true()@
      @even.apply(succ(n)) = not(even.apply(n))@
    \end{lstlisting}
  \end{block}

\end{frame}

\begin{frame}
  {(De|Re)functionalization}

\begin{block}{(Co)Data Fragments (\citeauthor{rendel15automatic})}
  \textbf{Data Fragment}: only data types, \underline{limited} pattern matching

  <->

  \textbf{Codata Fragment}: only codata types, \underline{limited} copattern matching
\end{block}

De- and refunctionalization are \textbf{symmetric} on these fragments.

\end{frame}

\begin{frame}
  {(De|Re)functionalization}

  \begin{block}{Defunctionalization (data/codata)}
    From \textbf{Codata Fragment} to \textbf{Data Fragment} programs.

    Replaces each \textbf{codata type def.} with a \textbf{data type def.}:

    \textbf{One constructor per function definition}.

    \textbf{One function def. per destructor}.
  \end{block}

  \begin{block}{Refunctionalization (data/codata)}
    From \textbf{Data Fragment} to \textbf{Codata Fragment} programs.

    Replaces each \textbf{data type def.} with a \textbf{codata type def.}:

    \textbf{One destructor per function definition}.

    \textbf{One function def. per constructor}.
  \end{block}
\end{frame}

\begin{frame}[fragile, shrink]
  {(De|Re)functionalization}

    \begin{block}{}
    \begin{lstlisting}[escapechar=!]
    !\tikzmark{nat1}!data Nat where
      !\tikzmark{zero1}!zero(): Nat
      !\tikzmark{succ1}!succ(Nat): Nat
    
    !\tikzmark{add1}!function add(Nat, Nat): Nat where
      add(zero(), x)  = x
      add(succ(x), y) = succ(add(x, y))
    \end{lstlisting}
    \end{block}
    \begin{block}{}
    \begin{lstlisting}[escapechar=!]
    !\tikzmark{nat2}!codata Nat where
      !\tikzmark{add2}!Nat.add(Nat): Nat
    
    !\tikzmark{zero2}!function zero(): Nat where
      zero().add(x)  = x

    !\tikzmark{succ2}!function succ(Nat): Nat where
      succ(x).add(y) = succ(x.add(y))
    \end{lstlisting}
    \end{block}

    \foreach[count=\i from 2] \x/\y in {nat1/nat2, zero1/zero2, succ1/succ2, add1/add2}
    {
      \redbox{\x}{\i}
      \redbox{\y}{\i}
    }
\end{frame}

\section
  {Transformations for Uroboro}

\againframe<4>{title}

\begin{frame}<1>[label=contribs]
  {Transformations for Uroboro}

  \underline{\textbf{Contributions of this work:}}

  \begin{itemize}
  \item \alert<2>{Formally defines the language \textbf{Uroboro}: \textbf{a common superset} of both the Data and Codata Fragment}

  \item \alert<3>{Extends de- and refunctionalization to all of Uroboro.}

  \item \alert<4>{Identifies a generalization for certain steps in these transformations, called \textbf{Extraction}.}

  \item \alert<5>{By-product: Shines some light on an asymmetry between patterns and copatterns.}
   \end{itemize}
\end{frame}

\againframe<2>{contribs}

\begin{frame}[fragile]
  {Transformations for Uroboro: Uroboro}

  \begin{block}{Uroboro example w/ patterns \underline{and} copatterns in one lhs}
    \begin{lstlisting}[style=base, gobble=4]
    data Expr where
      var(Nat): Expr
      fun(Var, Expr): Expr

    codata Val where
      Val.apply(Val): Val

    function eval(Expr, Env): Val where
      ...
      |eval(fun(var,expr), env).apply(val)| = 
        eval(expr, env.extendWith(var, val))
    \end{lstlisting}
  \end{block}

  Note: equations are \textbf{not ordered}
\end{frame}

\begin{frame}
  {Transformations for Uroboro: Uroboro}

   \begin{block}{Uroboro value judgement}
  \tiny{
  \begin{prooftree}
\AxiomC{$fun(t_1, ..., t_n) \neq^? q ~ \forall (q, t) \in \textrm{Rules}(prg)$}
\AxiomC{$\vdash_v t_1$}
\AxiomC{...}
\AxiomC{$\vdash_v t_n$}
\RightLabel{V\textsubscript{CodTFun}}
\QuaternaryInfC{$\vdash_v fun(t_1, ..., t_n)$, if $fun(t_1, ..., t_n)$ has codata type}
\end{prooftree}

\begin{prooftree}
\AxiomC{$t_0.des(t_1, ..., t_n) \neq^? q ~ \forall (q, t) \in \textrm{Rules}(prg)$}
\AxiomC{$\vdash_v t_0$}
\AxiomC{$\vdash_v t_1$}
\AxiomC{...}
\AxiomC{$\vdash_v t_n$}
\RightLabel{V\textsubscript{CodTDes}}
\QuinaryInfC{$\vdash_v t_0.des(t_1, ..., t_n)$, , if $t_0.des(t_1, ..., t_n)$ has codata type}
\end{prooftree}

\begin{prooftree}
\AxiomC{$\vdash_v t_1$}
\AxiomC{...}
\AxiomC{$\vdash_v t_n$}
\RightLabel{V\textsubscript{Con}}
\TrinaryInfC{$\vdash_v con(t_1, ..., t_n)$}
\end{prooftree}
}
  \end{block}

... more in chapter 2 of the thesis ...
\end{frame}

\againframe<3>{contribs}

\begin{frame}[fragile]
  {Transformations for Uroboro: De- and refunctionalization}

  \begin{block}{Two phases for de- and refunctionalization}
    \begin{itemize}
      \item \textbf{Unnesting}: brings (co)patterns to the limited form allowed in the (Co)Data Fragment
      \item \textbf{Core (de|re)func.:} essentially the (de|re)functionalization for the (Co)Data Fragment
    \end{itemize}
  \end{block}

  \textbf{Unnesting}: adapted for Uroboro from \citet{setzer14unnesting}
\end{frame}

\begin{frame}[fragile]
  {Transformations for Uroboro: De- and refunctionalization}

  \begin{block}{Uroboro example, unnested}
    \begin{lstlisting}[style=base, gobble=4]
    ...

    function eval(Expr, Env): Val where
      ...
      eval(fun(var,expr), env) =
        aux(var, expr, env)
        

    |function aux(Expr, Expr, Env): Val| where
      aux(var, expr, env).apply(val) =
        eval(expr, env.extendWith(var, val))
    \end{lstlisting}
  \end{block}

  Call this: \textbf{Extraction} of a destructor
\end{frame}

\begin{frame}
  {Transformations for Uroboro: De- and refunctionalization}

\underline{In general:}

Unnesting works on the coverage trees and ``undoes'' the coverage steps:
\begin{itemize}
\item Variable split: undone by constructor extraction

Ex.:

\only<-2>{$\substack{\texttt{f(\alert<2>{c1()}) = t1} \\ \texttt{f(\alert<2>{c2()}) = t2}}$}
\only<3->{\texttt{f(\alert<3>{x}) = \alert<3>{aux(x)}} | \alert<3>{$\substack{\texttt{aux(c1()) = t1} \\ \texttt{aux(c2()) = t2}}$}}

\item Result split: undone by destructor extraction

Ex.:

\only<-5>{$\substack{\texttt{\alert<5>{f(x).d1()} = t1} \\ \texttt{\alert<5>{f(x).d2()} = t2}}$}
\only<6-7>{\texttt{\alert<6>{f(x)} = \alert<6>{aux(x)}} | \alert<6>{$\substack{\texttt{aux(x).d1() = t1} \\ \texttt{aux(x).d2() = t2}}$}}

(other ex. on previous slide)
\end{itemize}

\end{frame}

\begin{frame}
  {Transformations for Uroboro: De- and refunctionalization}

\textbf{Important:} The extractions must be carried out in exactly the reverse order of the coverage derivation---otherwise semantics preservation is not guaranteed.

\end{frame}

\begin{frame}[fragile]
  {Transformations for Uroboro: De- and refunctionalization}

  \begin{block}{Unnesting order example}
    \begin{block}{}
      \begin{lstlisting}[style=base, gobble=4]
         function fun(S): T where
           |fun(c1()).d1()| = t1
           |fun(c1()).d2()| = t2
           fun(c2())      = t3
      \end{lstlisting}
    \end{block}
    Clearly, last coverage derivation step was the result split; thus extract destr. first!
    \begin{block}{}
      \begin{lstlisting}[style=base, gobble=4]
         function fun(S): T where
           |fun(c1())| = |aux()|
           fun(c2()) = t3
      \end{lstlisting}
    \end{block}
  \end{block}
\end{frame}

\begin{frame}[fragile]
  {Transformations for Uroboro: De- and refunctionalization}

  \begin{block}{Unnesting order example (cont'd)}
    First extracting the constructor changes the semantics!
    \begin{block}{}
      \begin{lstlisting}
         function fun(S): T where
           fun(x).d1() = ...
           fun(x).d2() = ...
           fun(x)      = ...
      \end{lstlisting}
    \end{block}
    The equations overlap, unlike the original equations.
  \end{block}
\end{frame}

\begin{frame}[fragile]
  {Transformations for Uroboro: De- and refunctionalization}
The unnested program can now be passed to core defunc. ...

  \begin{block}{Uroboro example, defunc'ed}
    \begin{lstlisting}[style=base, gobble=4]
    |data Val| where
      |aux(Expr, Expr, Env): Val|

    function eval(Expr, Env): Val where
      ...
      eval(fun(var,expr), env) =
        aux(var, expr, env)        

    |function apply(Val, Val): Val| where
      apply(aux(var, expr, env), val) =
        eval(expr, env.extendWith(var, val))
    \end{lstlisting}
  \end{block}
\end{frame}

\begin{frame}[fragile]
  {Transformations for Uroboro: De- and refunctionalization}
... or to core refunc.

  \begin{block}{Uroboro example, refunc'ed}
    \begin{lstlisting}[style=base, gobble=4]
    |codata Expr where|
      |Expr.eval(Env): Val|

    |function var(Nat): Expr| where
      ...

    |function fun(Var, Expr): Expr| where     
      fun(var, expr).eval(env) =
        aux(var, expr, env)

    ...
    \end{lstlisting}
  \end{block}
\end{frame}

\againframe<4>{contribs}

\begin{frame}
  {Transformations for Uroboro: Extraction}

  \begin{itemize}
    \item General intuition: \textbf{Extractions} make a program simpler \textbf{syntactically}, while preserving its \textbf{semantics} (cf. refactoring).

    \item In Uroboro: Can either extract destructors or constructors

    \item Metric for simplicity: maximal number of (co)patterns in any lhs

    \item Semantics preservation proof: Adapted and significantly extended from \citet{setzer14unnesting}
  \end{itemize}

\end{frame}

\againframe<5>{contribs}

\begin{frame}
  {Transformations for Uroboro: An asymmetry}
  \begin{itemize}
    \item In \citeauthor{rendel15automatic}'s Data Fragment: constructors only allowed in the first argument.

      not allowed: \texttt{f(c(), c'())}

      allowed: \texttt{f(c(), x)}

  \item Problem: No analogous limitation for copatterns (in their current form) possible!

      \texttt{f().d()}
  \end{itemize}
\end{frame}

\begin{frame}
  {Transformations for Uroboro: An asymmetry}
  \begin{itemize}
    \item Underlying problem: Copatterns allow to observe output usage in \textbf{only one dimension}, unlike patterns.

    \item Compare: the difference between natural deduction and arbitrary sequent calculus, which allows multiple formulas on right-hand sides.
  \end{itemize}
\end{frame}

\section{Conclusion}

\begin{frame}
  {Conclusion}
  \begin{itemize}
    \item \citet{rendel15automatic} made de- and refunctionalization symmetric on two fragment languages
    \item This work: makes them more practical by extending them to a full language (= common generalization of the fragments)
    \item Adapted the unnesting algorithm of \citet{setzer14unnesting} to achieve this
    \item Identified and proved correct a generalization of the unnesting steps
    \item Recognized an asymmetry between patterns and copatterns
  \end{itemize}

  \centering\textbf{Thanks!}
\end{frame}

\begin{frame}%[allowframebreaks]
  {Bibliography}
  \bibliographystyle{plainnat}
  \tiny\bibliography{bibliography}
\end{frame}

\end{document}

