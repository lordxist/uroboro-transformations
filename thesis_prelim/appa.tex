\chapter{Proofs}

\section{Proofs for section 3.3.2}

\begin{lemma}
\label{lem:app1}
Let $a, b, c$ be terms and $\mathcal{E}^{aux}$ be an evaluation context for $\longrightarrow^{aux}$. When
\[
\mathcal{E}^{aux}[a] \mapsto' b \land a \mapsto^{aux} c,
\]
then there is a $d$ such that
\[
\mathcal{E}^{aux}[c] \longrightarrow' d \land b {\longrightarrow^{aux}}^* d
\]
\begin{proof}
Since $a$ matches $q_\epsilon$, we know that $a$ is a function or destructor call, and not a constructor call. Because $\mathcal{E}^{aux}[a] = q_r[\sigma], b = t_r[\sigma]$ for an equation $r$, and $q_r$ and $q_{r'}$ don't overlap, it follows that $a$ is a subterm of a right-hand side of $\sigma$. Put in another way, there is a variable $x$ and a term $t(a)$ with subterm $a$ such that $\{x \mapsto t\} \subseteq \sigma$. Define a $\sigma_{mod}$ with $dom(\sigma_{mod}) = \sigma$ as $\sigma_{mod}(x) = t(c)$ and $\sigma_{mod}(y) = \sigma(y)$ for $y \neq x$, and choose $d = t_r[\sigma_{mod}]$. The two desired reductions can be derived as follows.
\begin{enumerate}
\item By the ``Subst'' rule for $r$:
\[
\mathcal{E}^{aux}[c] = q_r[\sigma_{mod}] \longrightarrow' t_r[\sigma_{mod}] = d
\]
This is possible because each immediate subterm $s$ of $\mathcal{E}^{aux}[a]$ is a value under judgement $\vdash'_v$, and it is $\langle \mathcal{E}^{aux}[a] \rangle^{aux^{-1}} = \langle \mathcal{E}^{aux}[c] \rangle^{aux^{-1}}$ since $\langle a \rangle^{aux^{-1}} = \langle c \rangle^{aux^{-1}}$, thus it is $\langle s \rangle^{aux^{-1}} = \langle s' \rangle^{aux^{-1}}$ for the subterm $s'$ of $\mathcal{E}^{aux}[c]$ corresponding to $s$ and consequently it is $\vdash'_v s'$, as well.

\item Define a $\sigma_{[]}$ with $dom(\sigma_{[]}) = \sigma$ as $\sigma_{[]}(x) = []$ and $\sigma_{[]}(y) = \sigma(y)$ for $y \neq x$. This way, $t_r[\sigma_{[]}]$ is a multi-hole context, such that $b = t_r[\sigma] = t_r[\sigma_{[]}][t(a), ..., t(a)]$ and $t_r[\sigma_{[]}][t(c), ..., t(c)] = t_r[\sigma_{mod}] = d$. By consecutive applications of the ``Cong'' rule on top of the ``Subst'' rule for $\epsilon$ we have the desired reduction sequence:
\[
b = t_r[\sigma_{[]}][t(a), ..., t(a)] \longrightarrow^{aux} t_r[\sigma_{[]}][t(c), t(a), ..., t(a)] \longrightarrow^{aux} t_r[\sigma_{[]}][t(c), ..., t(c)] = t_r[\sigma_{mod}] = d.
\]
\end{enumerate}
\end{proof}
\end{lemma}

\begin{fact}
\label{fac:app1}
For any term $a$ that is not a constructor call, pattern $p$, substitution $\sigma$ with a domain including at least the variables in $p$, and evaluation context $\mathcal{E}' \in \mathbf{EC}[\vdash'_v]$: When $\mathcal{E}'[a] = \sigma(p)$, then there is a variable $x$ in $p$ and an evaluation context $\mathcal{E}'_0 \in \mathbf{EC}[\vdash'_v]$ such that $\mathcal{E}'_0[a] = \sigma(x)$.
\begin{proof}
By induction on the structure of $p$.
\begin{itemize}
\item $p= y$ for some variable $y$.
Then it is $y = x$ and $\mathcal{E}'[a] = \sigma(x)$. Thus simply choose $\mathcal{E}'_0 = \mathcal{E}'$.

\item $p = con(p_1, ..., p_n)$ for some $n$-ary constructor $con$ and some patterns $p_1, ..., p_n$.
Then it is $\mathcal{E}'[a] = con(p_1, ..., p_n)[\sigma]$. When $\mathcal{E}' = []$, it would be $a = con(p^i_1, ..., p^i_{i_n})[\sigma]$, contradicting the premiss that $a$ isn't a constructor call. Thus it must be $\mathcal{E}' = con(\sigma(p_1), ..., \mathcal{E}'_{0'} ..., \sigma(p_n))$ for some $\mathcal{E}'_{0'} \in \mathcal{EC}[\vdash'_v]$, with $\mathcal{E}'_{0'}[a] = p_j$ for a $j \in \{1, ..., n\}$. By the induction hypothesis, there is a variable $x$ in $p_j$ and thus in $p$ such that $\sigma(x) = \mathcal{E}'_1$ for some $\mathcal{E}'_1 \in \mathbf{EC}[\vdash'_v]$.
\end{itemize}
\end{proof}
\end{fact}

\begin{lemma}
\label{lem:app2}
Let $a, b, c$ be terms and $\mathcal{E}' \in \textrm{EC}[\vdash'_v]$ an evaluation context. When
\[
a \longrightarrow' b \land \mathcal{E}'[a] \longrightarrow^{aux} c
\]
and both reduction steps are derived by the ``Subst'' rule, then there is a $d$ such that
\[
c {\longrightarrow'}^= d \land \mathcal{E}'[b] {\longrightarrow^{aux}}^= d.
\]
\begin{proof}
We distinguish the following cases for the form of $\mathcal{E}'$. (Note that only the third case, concerning function calls, is truly important, as the others are either essentially identical to it or trivial.)

\begin{enumerate}
\item $\mathcal{E}' = []$. Then it is $\mathcal{E}'[a] = a$. Since there are no overlapping equations, it follows that $b = c$. It also is $\mathcal{E}'[b] = b$. Thus, simply choose $d = b = c$.

\item $\mathcal{E}' = con(t_1, ..., t_{i-1}, \mathcal{E}'_0, t_{i+1}, ..., t_n)$, for some constructor $con$. In this case, $\mathcal{E}'[a]$ wouldn't match any lhs, contrary to assumption.

\item $\mathcal{E}' = fun(t_1, ..., t_{i-1}, \mathcal{E}'_0, t_{i+1}, ..., t_n)$, for some $\mathcal{E}'_0 \in \mathbf{EC}[\vdash'_v]$.
Because $\mathcal{E}'[a] = q_\epsilon[\sigma]$ for some $\sigma$, it is $q_\epsilon = fun(p_1, ..., p_n)$ with $\sigma(p_i) = \mathcal{E}'_0[a], \sigma(p_j) = t_j$ for $j \neq i$. For $j < i$, $t_j = \sigma(p_j)$ is a value w.r.t. $\vdash'_v$. By the definition of the value judgement, since these are subterms of the respective $t_j$, we know that the same holds for $\sigma(x)$, for any variable $x$ in one of the $p_j$ with $j < i$. From \autoref{fac:app1} we know that, since $a$ matches a lhs and is thus not a constructor call, it is $\sigma(x) = \mathcal{E}'_1[a]$ for an $\mathcal{E}'_1 \in \mathbf{EC}[\vdash'_v]$ and a variable $x$ in $p_i$. Thus we can choose $d = aux(\sigma(x_1), ..., \mathcal{E}'_1[b], ..., \sigma(x_k))$ and derive the first desired reduction as follows. By the definition of extractions, it is $t_\epsilon = aux(x_1, ..., x_k)$, with $x_1, ..., x_k$ the variables in $q_\epsilon$, in the order appearing there. Especially, it is $x_j = x$, for some $j \in \{1, ..., k\}$, and each variable $x_1, ..., x_{j-1}$ is contained in one of $p_1, ..., p_{i-1}$, and thus, as shown above, the $\sigma(x_1), ..., \sigma(x_{j-1})$ are all values w.r.t. $\vdash'_v$. Consequently, it is $c = t_\epsilon[\sigma] = aux(\sigma(x_1), ..., \sigma(x), ..., \sigma(x_k)) = aux(\sigma(x_1), ..., \mathcal{E}'_1[a], ..., \sigma(x_k)) \in \mathbf{EC}[\vdash'_v]$ and thus we can derive:

\begin{prooftree}
\AxiomC{(by the premiss)}
\UnaryInfC{$a \longrightarrow' b$}
\RightLabel{Cong}
\UnaryInfC{$aux(\sigma(x_1), ..., \mathcal{E}'_1[a], ..., \sigma(x_k)) \longrightarrow' d$}
\end{prooftree}

For the second reduction, let $\sigma^{aux} = \sigma \mid_{dom(\sigma) \setminus \{x\}} \cup \{x \mapsto \mathcal{E}'_0[b]\}$. Especially, this means that $\mathcal{E}'[b] = fun(\sigma(p_1), ..., \mathcal{E}'_1[b], ..., \sigma(p_n)) =^? q_\epsilon \searrow \sigma^{aux}$. Thus, apply the ``Subst'' rule for $\epsilon$ to get the reduction
\[
\mathcal{E}'[b] \longrightarrow^{aux} t_\epsilon[\sigma^{aux}].
\]
This is the desired reduction since
\[
t_\epsilon[\sigma^{aux}] = aux(x_1, ..., x_k)[\sigma^{aux}] = aux(\sigma(x_1), ..., \mathcal{E}'_1[b], ..., \sigma(x_k)) = d.
\]

\item $\mathcal{E}' = v.des(\overline{v}, \mathcal{E}'_0, \overline{t})$. The proof proceeds essentially identical to that for the function call case above; to see this, keep in mind that the destructor call syntax desugars to $des(v, \overline{v}, \mathcal{E}'_0, \overline{t})$. Thus the destructed value is actually just the first argument of the destructor, with the only difference being that it isn't an instance of a pattern. But this doesn't actually require a different argument than that of the previous function call case, because, regarding the arguments left of $\mathcal{E}'_0$, this argument only depends on them being values.

\item $\mathcal{E}' = \mathcal{E}'_0.des(\overline{t})$, with $\mathcal{E}'_0 \in \mathbf{EC}[\vdash'_v]$. By expanding along the definition of evaluation contexts, we know that $\mathcal{E}'$ either has the form (a) $fun(\overline{v}, \mathcal{E}'_0, \overline{t})\overline{des(\overline{t'})}$, or (b) $v.des(\overline{v}, \mathcal{E}'_0, \overline{t}).\overline{des(\overline{t'})}$. In case (a), proceed like in the function call case above, and just add the destructor calls behind the function calls to $fun$. In case (b), proceed like in the destructor call case above, and again just add the destructor calls behind the function calls to $fun$.
\end{enumerate}
\end{proof}
\end{lemma}

\cdpaux*
\begin{proof}
We make use of the fact that a reduction that can be derived by multiple applications of the ``Cong" rule can also be derived by just one application of the ``Cong" rule. When a reduction is derived by the ``Subst'' rule, it can also be derived by applying the ``Cong'' rule for the empty evaluation context on top of that. Thus, we have the following:
\begin{itemize}
\item From $a \longrightarrow' b$: $a = \mathcal{E}'[a'_0], b = \mathcal{E}'[b_0]$ for an $\mathcal{E} \in \mathbf{EC}[\vdash'_v]$ and terms $a'_0, b_0$ with $a'_0 \longrightarrow' b_0$ and $a'_0 = q_r[\sigma'], b_0 = t_r[\sigma']$ for some equation $r$.

\item From $a \longrightarrow^{aux} c$: $a = \mathcal{E}^{aux}[a^{aux}_0], c = \mathcal{E}^{aux}[c^0]$ for some term with hole $\mathcal{E}^{aux}$ and terms $a^{aux}_0, c^0$ with $a^{aux}_0 \longrightarrow^{aux} c^0$ and $a^{aux}_0 = q_\epsilon[\sigma^{aux}], c^0 = t_\epsilon[\sigma^{aux}]$.
\end{itemize}

We compare $\mathcal{E}'$ and $\mathcal{E}^{aux}$ and distinguish four cases.
\begin{itemize}
\item $\mathcal{E}' = \mathcal{E}^{aux}$. Then it is $a'_0 = a^{aux}_0$; since $\langle prg \rangle$ has no overlapping lhss, it must be $b^0 = c^0$, thus simply choose $d = b = c$.

\item The hole of $\mathcal{E}'$ properly contains that of $\mathcal{E}^{aux}$, i.e., there is a context $\mathcal{E}^{aux}_0 \neq []$ such that $\mathcal{E}^{aux} = \mathcal{E}'[\mathcal{E}^{aux}_0]$. It follows that $\mathcal{E}^{aux}_0[a^{aux}_0] = a'_0$. By \autoref{lem:app1} we have a $d_0$ with $\mathcal{E}^{aux}_0[c_0] \longrightarrow' d_0$ and $b_0 {\longrightarrow^{aux}}^* d_0$. That second reduction means that there are terms $t_1, ..., t_n$ with $b_0 \longrightarrow^{aux} t_1 \longrightarrow^{aux} ... \longrightarrow^{aux} t_n \longrightarrow^{aux} d_0$. Choose $d = \mathcal{E}'[d_0]$; the property holds since $b = \mathcal{E}'[b_0] \longrightarrow^{aux} \mathcal{E}'[t_1] \longrightarrow^{aux} ... \longrightarrow^{aux} \mathcal{E}'[t_n] \longrightarrow^{aux} \mathcal{E}'[d_0] = d$ and $c = \mathcal{E}^{aux}[c^0] = \mathcal{E}'[\mathcal{E}^{aux}_0][c^0] = \mathcal{E}'[\mathcal{E}^{aux}_0[c^0]] \longrightarrow' \mathcal{E}'[d_0] = d$.

\item The hole of $\mathcal{E}^{aux}$ properly contains that of $\mathcal{E}'$, i.e., there is a context $\mathcal{C} \neq []$ such that $\mathcal{E}' = \mathcal{E}^{aux}[\mathcal{C}]$. By fact \autoref{fac:chp21}, it follows that $\mathcal{E}^{aux} \in \textrm{EC}[\vdash'_v]$. Further, it follows that $\mathcal{C}[a'_0] = a^{aux}_0$. By \autoref{lem:app2} we have a $d_0$ with $c_0 \longrightarrow' d_0$ and $\mathcal{C}[b_0] \longrightarrow^{aux} d_0$. Choose $d = \mathcal{E}^{aux}[d_0]$; the property holds since $b = \mathcal{E}'[b_0] = \mathcal{E}^{aux}[\mathcal{C}][b_0] = \mathcal{E}^{aux}[\mathcal{C}[b_0]] \longrightarrow^{aux} \mathcal{E}^{aux}[d_0] = d$ and $c = \mathcal{E}^{aux}[c_0] \longrightarrow' \mathcal{E}^{aux}[d_0] = d$.

\item Neither the hole of $\mathcal{E}'$ contains that of $\mathcal{E}^{aux}$, nor the other way around. Then we have a two-hole context $\mathcal{E}^{merge}$ such that $\mathcal{E}^{merge}[[], a^{aux}_0] = \mathcal{E}'$ and $\mathcal{E}^{merge}[a'_0, []] = \mathcal{E}^{aux}$. Choose $d$ as $\mathcal{E}^{merge}[b_0, c_0]$. The property holds because $b = \mathcal{E}'[b_0] = \mathcal{E}^{merge}[[], a^{aux}_0][b_0] = \mathcal{E}^{merge}[b_0, a^{aux}_0] = \mathcal{E}^{merge}[b_0, []][a^{aux}_0] \longrightarrow^{aux} \mathcal{E}^{merge}[b_0, []][c_0] = \mathcal{E}^{merge}[b_0, c_0] = d$ and $c = \mathcal{E}^{aux}[c_0] = \mathcal{E}^{merge}[a'_0, []][c_0] = \mathcal{E}^{merge}[a'_0, c_0] = \mathcal{E}^{merge}[[], c_0][a'_0] \longrightarrow' \mathcal{E}^{merge}[[], c_0][b_0] = d$. This second reduction is possible because it is $\mathcal{E}^{merge}[[], c_0] \in \mathbf{EC}[\vdash'_v]$, since $\mathcal{E}' = \mathcal{E}^{merge}[[], a^{aux}_0] \in \mathbf{EC}[\vdash'_v]$ and $\langle a^{aux}_0 \rangle^{aux^{-1}} = \langle c_0 \rangle^{aux^{-1}}$ and thus $\vdash'_v a^{aux}_0$ iff $\vdash'_v c_0$.
\end{itemize}

\end{proof}

\begin{lemma}
\label{lem:app3}

For a term $t$ with $\vdash'_v t$ and $\not\vdash_v t$, there is a term $t'$ and reductions
\[
t \longrightarrow t'
\]
and
\[
t \longrightarrow^{aux} t'.
\]

\begin{proof}

By induction on the structure of $t$. We distinguish two cases.
\begin{enumerate}
\item All immediate subterms of $t$ are values under judgement $\vdash_v$. We will show that $t$ matches $q_\epsilon$, and from this the desired reductions immediately follow. Because $\not\vdash_v t$, by the definition of the value judgement $t$ itself has to match some lhs $q$ of $\langle prg \rangle$. Since $\vdash'_v t$, there cannot be an equation $q'$ of $prg$ with $\langle q \rangle^{aux^{-1}} = q'$. By the definition of extractions, the only equation of $\langle prg \rangle$ for which this can be the case is $q = q_\epsilon$. Consequently, we have some $\sigma$ that substitutes only with terms which are values under judgement $\vdash_v$ and the desired reductions
\[
t = q_\epsilon[\sigma] \longrightarrow t_\epsilon[\sigma], t = q_\epsilon[\sigma] \longrightarrow^{aux} t_\epsilon[\sigma].
\]

\item There are immediate subterms of $t$ which aren't values under judgement $\vdash_v$.
Let $t_1, ..., t_n$ be the immediate subterms of $t$, and let $k$ be the smallest index for which $\not\vdash_v t_k$. Define $\mathcal{E} \in \mathbf{EC}$ by taking $t$ and replacing $t_k$ by a hole. By the induction hypothesis, we have a term $t'_k$ and reductions
\[
t_k \longrightarrow t'_k, t_k \longrightarrow^{aux} t'_k.
\]
Consequently, we have the desired term as $t' = \mathcal{E}[t'_k]$ and reductions
\[
t = \mathcal{E}[t_k] \longrightarrow \mathcal{E}[t'_k] = t', t = \mathcal{E}[t_k] \longrightarrow^{aux} \mathcal{E}[t'_k] = t'.
\]
\end{enumerate}

\end{proof}

\end{lemma}

\compl*
\begin{proof}

There is an evaluation context $\mathcal{E} \in \textrm{EC}[\vdash'_v]$, with $\mathcal{E} \not\in \textrm{EC}$, and terms $a^0, b^0$, such that $a = \mathcal{E}[a^0], b = \mathcal{E}[b^0]$. Because $\mathcal{E} \not\in \textrm{EC}$, there has to be a subterm $t$ of $\mathcal{E}$ with $\not\vdash_v t$. Construct an $\mathcal{E}' \in \mathbf{EC}$ from $\mathcal{E}$ by replacing $t$ with $[]$ and the original hole with $a^0$. By \autoref{lem:app3}, we have a term $t_{aux}$ and reductions
\[
t \longrightarrow t_{aux}, t \longrightarrow^{aux} t_{aux},
\]
and consequently there is a $c = \mathcal{E}'[t_{aux}]$ with
\[
a = \mathcal{E}'[t] \longrightarrow \mathcal{E}'[t_{aux}], a = \mathcal{E}'[t] \longrightarrow^{aux} \mathcal{E}'[t_{aux}].
\]

\end{proof}

\clearpage
\newpage
