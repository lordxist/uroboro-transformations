\chapter{Pattern alignment algorithm}

Here we define the algorithm that aligns patterns; we present a calculus for it. Afterwards we give a proof of its correctness.

We write $q \rhd Q \searrow \Sigma$ when the algorithm produces a result $\Sigma$ for a copattern $q$ and a copattern set $Q$ that $q$ is to be aligned with. This $\Sigma$ is a set of substitutions $\sigma$ such that $q[\sigma]$ is aligned with all elements of $Q$ and such that $\bigcup_{\sigma \in \Sigma} q[\sigma]$ is equivalent to $q$. Similarly, we write $q \rhd q' \searrow \Sigma$ for the sub-algorithm used for aligning $q$ with an individual copattern $q'$.

\begin{prooftree}
\AxiomC{}
\RightLabel{QA\textsubscript{Empty}}
\UnaryInfC{$q \rhd \emptyset \searrow \mathbf{1}$}
\end{prooftree}

\begin{prooftree}
\AxiomC{$q \rhd Q \searrow \Sigma$}
\AxiomC{$q \rhd q' \searrow \Sigma'$}
\RightLabel{QA\textsubscript{Head}}
\BinaryInfC{$q \rhd q' : Q \searrow \Sigma' \circ \Sigma$}
\end{prooftree}

\begin{prooftree}
\AxiomC{$\overline{p \rhd p' \searrow \Sigma}$}
\RightLabel{qA}
\UnaryInfC{$fun(\overline{p}) \rhd fun(\overline{p'}) \searrow \prod_i \Sigma_i$}
\end{prooftree}

$p \rhd p' \searrow \Sigma$ means that the sub-algorithm for pattern alignment, presented below, produces the set of substitutions $\Sigma$.

\begin{prooftree}
\AxiomC{}
\RightLabel{PA\textsubscript{VarVar}}
\UnaryInfC{$x \rhd y \searrow \mathbf{1}$}
\end{prooftree}

\begin{prooftree}
\AxiomC{$\overline{y^{\mathit{fresh}} \rhd p \searrow \Sigma}$}
\RightLabel{PA\textsubscript{VarCon}}
\UnaryInfC{$x \rhd con^\tau(\overline{p}) \searrow \{ x \mapsto con(\overline{\Sigma(y^{\mathit{fresh}})}) \} \oplus \bigoplus\limits_{\substack{con' \in \textrm{Cons}_\tau \\ con' \neq con}} x \mapsto con'(\overline{z^{\mathit{fresh}}_{con'}})$}
\end{prooftree}

\begin{prooftree}
\AxiomC{}
\RightLabel{PA\textsubscript{ConVar}}
\UnaryInfC{$con(\overline{p}) \rhd x \searrow \mathbf{1}$}
\end{prooftree}

\begin{prooftree}
\AxiomC{$\overline{p \rhd p' \searrow \Sigma}$}
\RightLabel{PA\textsubscript{ConCon\textsubscript{$=$}}}
\UnaryInfC{$con(\overline{p}) \rhd con(\overline{p'}) \searrow \prod_i \Sigma_i$}
\end{prooftree}

\begin{prooftree}
\AxiomC{$con \neq con'$}
\RightLabel{PA\textsubscript{ConCon\textsubscript{$\neq$}}}
\UnaryInfC{$con(\overline{p}) \rhd con'(\overline{p'}) \searrow \mathbf{1}$}
\end{prooftree}

Aligning of patterns is lifted to programs in the following, straightforward way; for this, let $Q_{def}$ denote the lhss of function definition $def$.
\begin{alignat*}{4}
\langle prg \rangle^{align\_patterns} & = &&\{ \textrm{\textbf{function }} fun(\tau_1, ..., \tau_n): \sigma \textrm{\textbf{ where }} \\
& &&\quad \{ `` q' = t " ~ | ~ q' \in \Sigma(q), q \rhd Q_{def} \searrow \Sigma, `` q = t " \in eqns \} \\
& && | ~ def = `` fun (\tau_1, ..., \tau_n): \sigma \textrm{\textbf{ where }} eqns " \in prg \} \\
& \cup && \{ def ~ | ~ def \in prg, def \textrm{ is (co)data def. } \} \span\span\span\span
\end{alignat*}

\section{Proof of correctness}

The following lemma captures that this algorithm can actually be used for its intended purpose.

\begin{lemma}
For any set of copatterns $Q$ and any two of its elements $q_1, q_2$ with $q_1 \rhd Q \searrow \Sigma_1, q_2 \rhd Q \searrow \Sigma_2$ for some $\Sigma_1, \Sigma_2$, all elements of $\Sigma_1(q_1)$ align with all elements of $\Sigma_1(q_2)$.
\end{lemma}

To prove this, we first show a lemma for the sub-algorithm for pattern alignment.

\begin{lemma}
For any two patterns $p_1, p_2$ with $p_1 \rhd p_2 \searrow \Sigma_1, p_2 \rhd p_1 \searrow \Sigma_2$ for some $\Sigma_1, \Sigma_2$, all elements of $\Sigma_1(p_1)$ align with all elements of $\Sigma_2(p_2)$.

\begin{proof}
By induction on the structure of $p_1$.

\underline{$p_1$ is a variable:} By induction on the structure of $p_2$.
\begin{itemize}
\item \underline{$p_2$ is a variable:} Then we know that the last steps in the derivations of both $p_1 \rhd p_2 \searrow \Sigma_1$ and $p_2 \rhd p_1 \searrow \Sigma_2$ use PA\textsubscript{VarVar}. It follows that the identity substitution is the only element of $\Sigma_1$ and the only element of $\Sigma_2$. Thus the only element of $\Sigma_1(p_1)$ is $p_1$ itself, and the only element of $\Sigma_2(p_2)$ is $p_2$ itself, and these two align since they are both variables.

\item \underline{$p_2 = con(\overline{p'})$:} Then we know that the last step in the derivation of $p_1 \rhd p_2 \searrow \Sigma_1$ uses PA\textsubscript{VarCon}, and that it is $\overline{y^{fresh} \rhd p' \searrow \Sigma'}$ such that $\Sigma_2 = \bigoplus_{con' \in \textrm{Cons}_\tau} x \mapsto con'(\overline{\Sigma'(y^{\mathit{fresh}})})$. By the induction hypothesis we have that all elements of $\Sigma'_i(y^{fresh}_i)$ align with all elements of $\Sigma''_i(p'_i)$, with $p'_i \rhd y^{fresh}_i \searrow \Sigma''$. The only element of $\Sigma''_i(p'_i)$ is $p'_i$ itself, since the last step in the derivation of $p'_i \rhd y^{fresh}_i \searrow \Sigma''$ uses PA\textsubscript{ConVar} or \textsubscript{VarVar}. Thus $p'_i$ aligns with all elements of $\Sigma'_i(y^{fresh}_i)$.

We also know that the last step in the derivation of $p_2 \rhd p_1 \searrow \Sigma_2$ uses PA\textsubscript{ConVar}, thus the only element of $\Sigma_2$ is the identity substitution, and therefore the only element of $\Sigma_2(p_2)$ is $p_2$ itself. All elements of $\Sigma_1(p_1)$ align with $p_2$, because they are either calls to constructors different from that of $p_2$, or calls to the constructor $con$ of $p_2$ and the respective arguments of the constructor calls of $p_1$ and $p_2$ align. This is because, as shown above, each $p'_i$ aligns with all elements of $\Sigma'_i(y^{fresh}_i)$.
\end{itemize}

\underline{$p_1 = con(\overline{p'})$:} By induction on the structure of $p_2$.
\begin{itemize}
\item \underline{$p_2$ is a variable:} The argument is the same as when $p_1$ was a variable and $p_2$ a constructor call, only with the roles of $p_1$ and $p_2$ reversed.

\item \underline{$p_2 = con'(\overline{p''})$:} We distinguish two cases.
\begin{itemize}
\item $con \neq con'$. Then we know that the last steps in the derivations of both $p_1 \rhd p_2 \searrow \Sigma_1$ and $p_2 \rhd p_1 \searrow \Sigma_2$ use PA\textsubscript{ConCon\textsubscript{$\neq$}}. It follows that the identity substitution is the only element of $\Sigma_1$ and the only element of $\Sigma_2$. Thus the only element of $\Sigma_1(p_1)$ is $p_1$ itself, and the only element of $\Sigma_2(p_2)$ is $p_2$ itself, and these two align since they are constructor calls of different constructors.

\item $con = con'$. Then we know that the last steps in the derivations of both $p_1 \rhd p_2 \searrow \Sigma_1$ and $p_2 \rhd p_1 \searrow \Sigma_2$ use PA\textsubscript{ConCon\textsubscript{$=$}}, and that it is $\overline{p' \rhd p'' \searrow \Sigma'}$ such that $\Sigma_1 = \prod_i \Sigma'_i$ and $\overline{p'' \rhd p' \searrow \Sigma''}$ such that $\Sigma_2 = \prod_i \Sigma''_i$. By the induction hypothesis we know that each element of $\Sigma'(p'_i)$ aligns with each element of $\Sigma''(p''_i)$. It follows that each element of $\Sigma_1(p_1)$ aligns with each element of $\Sigma_2(p_2)$.
\end{itemize}
\end{itemize}

\end{proof}
\end{lemma} 

Now, we get back to the proof of Lemma B.1.1.

\begin{proof}[Proof of Lemma B.1.1]
By induction on the derivation of $q_1 \rhd Q \searrow \Sigma_1$.

TODO
\end{proof}
