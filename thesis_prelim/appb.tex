\chapter{German translation of the abstract}

Rendel et al. präsentieren zwei funktionale Programmiersprachen auf denen Reynolds Defunktionalisierung beziehungsweise Danvys Refunktionalisierung total definiert sind. Diese Sprachen, genannt Codata Fragment und Data Fragment, unterstützen Abels Copattern Matching beziehungsweise das übliche Pattern Matching funktionaler Sprachen. Rendel et al. beabsichtigen, dass diese Sprachen Fragmente einer gemeinsamen Sprache mit dem Namen Uroboro sind, welche sowohl über Pattern als auch über Copattern Matching verfügt. In dieser Thesis definieren wir Uroboro formal und entwickeln automatische Programmtransformationen für diese Sprache. Wir erweitern Rendel et al.s automatische De- und Refunktionalisierung auf die gesamte Sprache Uroboro. Außerdem identifizieren wir eine Verallgemeinerung einiger der Schritte aus denen diese Transformationen aufgebaut sind, welche wir als ``Extraction'' bezeichnen. Als ein Nebenprodukt unserer Arbeit beleuchten wir eine Asymmetrie zwischen Pattern und Copattern Matching.
